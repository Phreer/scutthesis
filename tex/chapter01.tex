\chapter{绪论}
\section{引言}

当今社会,科技的飞速发展为大家􏲁供了快捷与舒适,但与此同时也增添了在信
息安全上的危险。在过去的二十几年来,我们通过数字密码来鉴别身份,但是随着科
技的发展,不法分子借用高科技犯罪的案例年年增高,密码被盗的情况时常发生。因
此,怎样科学准确的辨别每一个人的身份则成为当今社会的重要问题。

\section{研究背景}
随着科技的日益发展,传统的密码因为记忆的繁琐以及容易被盗,似乎已经不再 
能满足这个通信发达的社会的需求。人们急需一种更便捷而且辨识度更高的方式来辨 
识身份。循着便捷与辨识度高这两个约束条件 \cite{ELIDRISSI94},我们联想到
的便是存在于每个人身上的生物特征,所以基于每个人身上不同的生物特征而研究的
鉴别技术现在成为了身份辨别技术上的主流。

\section{研究现状}
笔迹 \cite{imgprocesszh} 获取的方式有两种,所以鉴别方式也分为离线鉴别
和在线鉴别 \cite{lecun1998gradient}。在线鉴别是采用专用的数字板来实时
收集书写信号。由文献\parencite{RManual,kocher99,chen2007ewi} 可知,
因为信号是实时采集的,所以能采集的数据不仅包括笔迹序列,而且可以采集到书写时
的加速度、压力、速度等丰富有用的动态信息。

\section{论文结构}
本文分为四章。其中第一章简述了笔迹识别的研究背景和意义以及笔迹识别的基
础知识等。第二章节从卷积神经网络的发展历史、网络结构、学习规律三方面详细的
讲述了卷积网络的基础知识。第三章针对本文中的手写数字及写字人实验具体设计卷
积神经网络的网络结构以及训练过程。第五章节是手写数字识别及写字人识别实验的
结果与分析。